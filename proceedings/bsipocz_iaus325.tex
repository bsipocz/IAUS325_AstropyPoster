% iaus2esa.tex -- sample pages for Proceedings IAU Symposium document class
% (based on v1.0 cca2esam.tex)
% v1.04 released 17 May 2004 by TechBooks
%% small changes and additions made by KAvdH/IAU 4 June 2004
% Copyright (2004) International Astronomical Union

\NeedsTeXFormat{LaTeX2e}

\documentclass{iau}
\usepackage{graphicx}

\title[The astropy project] %% give here short title %%
{The astropy project: A community Python library for astronomy}

\author[Brigitta Sip{\H o}cz on behalf of the Astropy Collaboration] %% give here short author list %%
{Brigitta Sip{\H o}cz on behalf of the Astropy Collaboration$^1$}

\affiliation{$^1$ email: {\tt bsipocz@gmail.com}}

\pubyear{2017}
\volume{325}  %% insert here IAU Symposium No.
\setcounter{page}{1}
\jname{Astroinformatics}
\editors{Massimo Brescia; S. George Djorgovski; Eric
  Feigelson;\\ Giuseppe Longo \& Stefano Cavuoti}
\begin{document}

\maketitle

\begin{abstract}
The Astropy Project is a community effort to develop a single core package
for Astronomy in Python and foster interoperability between Python Astronomy
packages, and is one of the largest open-source collaborations in
Astronomy. We present an overview of the project, provide an update on the
latest status of the core package, which saw the v1.2 release this June. We
describe the "affiliated packages": Python packages that use Astropy and are
associated with the project, but are not actually a part of the core library
itself. We also highlight the infrastructural tools we provide for these
packages.
\keywords{miscellaneous, methods: data analysis, methods: miscellaneous} % There aren't relevant keywords in keywords.tex...
\end{abstract}

\firstsection % if your document starts with a section,
              % remove some space above using this command.
\section{Introduction}

\section{Overview}

\end{document}
