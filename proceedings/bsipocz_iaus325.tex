% iaus2esa.tex -- sample pages for Proceedings IAU Symposium document class
% (based on v1.0 cca2esam.tex)
% v1.04 released 17 May 2004 by TechBooks
%% small changes and additions made by KAvdH/IAU 4 June 2004
% Copyright (2004) International Astronomical Union

\NeedsTeXFormat{LaTeX2e}

\documentclass{iau}
\usepackage{graphicx}

\title[The astropy project] %% give here short title %%
{The astropy project: A community Python library for astronomy}

\author[Brigitta Sip{\H o}cz on behalf of the Astropy Collaboration] %% give here short author list %%
{Brigitta Sip{\H o}cz on behalf of the Astropy Collaboration$^1$}

\affiliation{$^1$ email: {\tt bsipocz@gmail.com}}

\pubyear{2017}
\volume{325}  %% insert here IAU Symposium No.
\setcounter{page}{1}
\jname{Astroinformatics}
\editors{Massimo Brescia; S. George Djorgovski; Eric
  Feigelson;\\ Giuseppe Longo \& Stefano Cavuoti}
\begin{document}

\maketitle

\begin{abstract}
The Astropy Project is a community effort to develop a single core package
for Astronomy in Python and foster interoperability between Python Astronomy
packages, and is one of the largest open-source collaborations in
Astronomy. We present an overview of the project, provide an update on the
latest status of the core package, which saw the v1.2 release this June. We
describe the "affiliated packages": Python packages that use Astropy and are
associated with the project, but are not actually a part of the core library
itself. We also highlight the infrastructural tools we provide for these
packages.
\keywords{miscellaneous, methods: data analysis, methods: miscellaneous} % There aren't relevant keywords in keywords.tex...
\end{abstract}

\firstsection % if your document starts with a section,
              % remove some space above using this command.

\section{The astropy project}

Python and improve usability, foster interoperability, and collaboration
between Python astronomy packages.

The core astropy package contains functionality aimed at professional
astronomers and astrophysicists, but can be useful to anyone developing
astronomy software. The Astropy Project also includes affiliated packages,
Python packages that are not necessarily developed by the core development
team, but share the goals of Astropy, and often build from the core
package's code and infrastructure.

\section{astropy core functionality}

\begin{itemize}
  \item Astronomy computations and utilities
  \begin{itemize}
    \item \textsc{\texttt{astropy.cosmology}}\\
      Contains classes for representing cosmologies, and utility functions for
      calculating commonly used quantities that depend on a cosmological model
      including distances, ages and lookback times corresponding to a measured
      redshift or the transverse separation corresponding to a measured angular
      separation.
    \item \textsc{\texttt{astropy.convolution}}\\
      Provides convolution functions and kernels that offers improvements compared
      to the scipy.ndimage routines
    \item \textsc{\texttt{astropy.stats}}\\
      Holds statistical functions and algorithms frequently used in astronomy,
      some of which are more specialized than one would find in scipy.stats.
    \item \textsc{\texttt{astropy.visualization}}\\
      Provides functionality that can be helpful when visualizing data. At the
      moment, functionality include enhanced histograms, image normalizing
      (including both scaling and stretching), and custom plotting styles for
      matplotlib.
  \end{itemize}
  \item Core data structures and transformations
  \begin{itemize}
    \item \textsc{\texttt{astropy.units}}\\
      Handles defining, and converting between, and performing arithmetic with
      physical quantities (using a powerful Quantity class). It also handles
      logarithmic units such as magnitude and decibel.
    \item \textsc{\texttt{astropy.constants}}\\
      Contains a number of physical constants useful in astronomy and
      physics. Constants are represented as Quantity objects with additional
      metadata describing their provenance and uncertainties.
    \item \textsc{\texttt{astropy.nddata}}\\
      Provides an NDData class for managing n-dimensional datasets with support
      for attached metadata, errors, and masking.
    \item \textsc{\texttt{astropy.time}}\\
      Provides functionality for manipulating times and dates. Specific emphasis
      is placed on supporting time scales (e.g. UTC, TAI, UT1) and time
      representations (e.g. JD, MJD, ISO 8601) that are used in astronomy. Wraps
      the ERFA library of time and calendar routines, which is a fork of SOFA
      released under a less restrictive license.
    \item \textsc{\texttt{astropy.coordinates}}\\
      Provides classes for representing celestial/spatial coordinates and
      transformation functions for converting between standard systems in a
      uniform way.
    \item \textsc{\texttt{astropy.wcs}}\\
      Contains utilities for managing WCS transformations in FITS files. These
      transformations map the pixel locations in an image to their real-world
      units.
    \item \textsc{\texttt{astropy.modeling}}\\
      Provides a framework for representing models, performing model evaluation,
      and fitting. It supports 1D and 2D models and fitting with parameter
      constraints, as well as an easy to use API for defining new models and
      fitting algorithms.
    \item \textsc{\texttt{astropy.table}}\\
      Provides a powerful Table class for storing and manipulating heterogeneous
      tables of data in a way that is familiar to Numpy users, though with more
      flexibility than the data structures built into Numpy including support for
      metadata and masking.
  \end{itemize}
  \item Files and I/O
  \begin{itemize}
    \item \textsc{\texttt{astropy.io.fits}}\\
      Reading and writing FITS files including support for many non-standard FITS
      conventions.
    \item \textsc{\texttt{astropy.io.ascii}}\\
      Reading from and writing a wide range of plain text formats to and from
      Astropy Table objects.
    \item \textsc{\texttt{astropy.io.votable}}\\
      Validating, and reading and writing the IVOA VOTable format to and from
      Astropy Table objects.
  \end{itemize}
\end{itemize}

\section{Community}

The astropy project is made possible through the efforts of members of the
astronomy community. Over 160 individuals (including students, postdocs,
staff, and faculty) contributed to the core package as of the 1.2
release. astropy mentored 19 Google Summer of Code students over the last 4
years. The whole astropy project is overseen by the coordination committee.

\subsection{Contribution cycle}

astropy’s entire development process is hosted on GitHub. This enables the
contributors, most of whom have never met each other, to work together in
the best collaborative way. In addition to the source code repository,
GitHub provides a public issue tracker and other project management tools.

Code contributions are submitted via “pull requests (PR)” and should be
include unit tests and documentation. PRs are reviewed by at least one of
the maintainers of the package. Using continuous integration (CI) all PRs
are thoroughly tested on multiple platforms and with different versions of
Python and NumPy. Once the review process is finished and all tests pass,
PRs are merged and get included in the next appropriate release.


\section{Release cycle}

astropy has feature releases every 6 month, and between feature releases
additional bugfix releases that contain only bugfixes but no new
features. The current latest release is v1.2.1.  Long-term support (LTS)
releases continue to receive bugfixes for 2 years with no changes to the
API. They are ideal for pipelines and other applications where API stability
is essential. The latest LTS version is v1.0.10.  Package template astropy
provides a template package that any Python package is welcome to use (many
affiliated packages do so).  The template contains a ready-to-go package
layout. It provides infrastructure such as documentation tools, testing
framework and CI templates, configurations, Cython integration and
documentation on how to make it all work.


\section{Affiliated packages}

Affiliated packages contain functionality that is more specialized, or have
license incompatibility, or have enough external dependencies (e.g. GUI
libraries) that limit their integration possibility into the core and thus
they are more suitable to be separate packages. In general we hope that
becoming an affiliated package is seen as a good way for new and existing
packages to gain exposure.


\subsection{Package template}

astropy provides a template package that any Python package is welcome to
use (many affiliated packages do so).  The template contains a ready-to-go
package layout. It provides infrastructure such as documentation tools,
testing framework and CI templates, configurations, Cython integration and
documentation on how to make it all work.

\subsection{Subset of affiliated packages}:

\begin{itemize}
\item astroplan: Observation planning and scheduling
\item APLpy: Publication-quality image plotting
\item astroML: Machine learning, statistics and data mining
\item astroquery: Interface to many web services/archives
\item ccdproc: CCD image reduction
\item gammapy: Gamma-ray data analysis
\item galpy: Galactic Dynamics
\item halotools: Building galaxy-halo models \& analyzing dark matter halos
\item ginga: FITS viewer GUI
\item glue: Multidimensional data exploration
\item photutils: Source detection and photometry
\item reproject: Image reprojection/resampling
\item sncosmo: Supernova cosmology analysis
\item spectral-cube: Spectral cube data analysis
\end{itemize}

\section{References}
\subsection{Services and packages we use}

TODO: insert logos here.

\subsection{Citing astropy}

If you use astropy for work/research presented in a publication, we ask that
you cite the Astropy Paper: Astropy: A community Python package for
astronomy, Astropy Collaboration, A\&A, 2013

We provide the following as a standard acknowledgment you can use if there
is not a specific place to cite the paper: This research made use of
Astropy, a community developed core Python package for Astronomy (Astropy
Collaboration, 2013).

\end{document}
